%%%%%%%%%%%%%%%%%%%%%%%%%%%%%%%%%%%%%%%%%%%%%%%%%%%%%%%%%%%%%%%%%%%%%%%%%%%%%

\documentclass[12pt, a4paper, oneside]{article}

\input{packages.tex}
\usepackage{csquotes}
\input{macros.tex}

\title{Side-Channel Attack}
\author{
    Leonardo Costa Santos - 10783142 &&
    Lucas Paiolla Forastiere - 11221911 &&
    Julia Leite - 11221797
}
\date{\today}

%%%%%%%%%%%%%%%%%%%%%%%%%%%%%%%%%%%%%%%%%%%%%%%%%%%%%%%%%%%%%%%%%%%%%%%%%%%%%
%%%%%%%%%%%%%%%%%%%%%%%%%%%%%%%%%%%%%%%%%%%%%%%%%%%%%%%%%%%%%%%%%%%%%%%%%%%%%

\begin{document}

\maketitle
\newpage

\section{Introdução}

Side-Channel Attacks, que do inglês significa, Ataque por Canal Lateral são
jeitos explorar vunerabilidades físicas de componentes eletrônicos, como a CPU
de um computador.

O nome vem do fato deles não atacarem "pela porta da frente", mas sim algum
"rastro" físico que um componente deixa ao fazer determinadas ações.

Um SCA não necessariamente tem a ver com componentes eletrônicos, pois podemos,
por exemplo, descifrar uma senha de alguém captando os sons do teclado. Em
geral, o SCA ataca um ponto fraco de um componente que não tem nada a ver com o
seu funcionamento em si (como no exemplo do teclado, o teclado teoricamente não
tem a responsabilidade de deixar os barulhos de cada tecla iguaizinhos). Daí
então o nome \textit{canal lateral}.
% TODO: Explicar melhor o que é side channel <18-11-20, Lucas> %

Os dois principais SCA, que tornaram o ``ramo'' famoso foram o \textit{Meltdown}
e o \textit{Spectre}, descobertos por independentemente por uma série de
pesquisadores, mas destacando-se o grupo Project Zero da
Google~\cite{Lipp2018meltdown}.

Entretanto, existem muitas classes de SCA, como ataques ao cache (que é o caso
dos dois exemplos citados), ataques que monitoram a energia consumida pelo
computador, ataques que monitoram o eletromagnetismo eminido, ataques que
monitoram o som emitido (como o exemplo do teclado), ataques que recuperam dados
excluidos do disco entre muitos outros.

\newpage

%%%%%%%%%%%%%%%%%%%%%%%%%%%%%%%%%%%%%%%%%%%%%%%%%%%%%%%%%%%%%%%%%%%%%%%%%%%%%

\section{História} % julia

Primeira pessoa a usar o termo (Side Channel Cryptanalysis of Product Ciphers)
Primeiro ataque descoberto (TEMPEST 1942)
Popularização do termo
Mais descobertas de ataques

\newpage

%%%%%%%%%%%%%%%%%%%%%%%%%%%%%%%%%%%%%%%%%%%%%%%%%%%%%%%%%%%%%%%%%%%%%%%%%%%%%

\section{Classificações de Side-Channel Attacks} % lucas


\subsection{Classificação por (?)} % se é passivo ou ativo

\subsection{Classificação por grau da invasão} % residual integrity (melhorar dps)

%%%%%%%%%%%%%%%%%%%%%%%%%%%%%%%%%%%%%%%%%%%%%%%%%%%%%%%%%%%%%%%%%%%%%%%%%%%%%

\section{Exemplos de Side-Channel Attacks}

\subsection{Meltdown} % lucas

Teste citando\cite{Lipp2018meltdown}
\subsection{Spectre} % leo
\subsection{CacheOut} % leo
CacheOut é um ataque de \emph{Microarchitectural Data Sampling} (MDS), ou
Amostragem de Dados Microarquitetural, capaz de evitar as medidas de segurança
contra MDSs dos processadores da Intel. Este ataque é capaz de vazar dados
atravéz de barreiras de segurança como o isomento de mémoria entre processos,
entre \emph{user/kernel spaces}, \emph{SGX enclaves} e máquinas virtuais.
Ao causar contenção de linhas de cache, CacheOut causa a expulsão de dados do
cache e lê estes dados dos LFBs com um ataque TAA, conseguindo vazar páginas
inteiras de memória.~\cite{schaik2020cacheout}

\subsubsection{Line Fill Buffers}
\emph{Line Fill Buffers} (LFBs) são \emph{buffers} microarquiteturais usados
para armazenar dados durante acessos ao cache L1, tratando de pedidos ao
cache em \emph{cache misses} e temporariamente armazenando dados em acessos
à memória e operações de I/O. Também podem ser usados em \emph{cache hits}
e para encaminhar dados para operações de leitura e escrita no
cache.~\cite{IntelMDS}

\subsubsection{Transactional Synchronization Extensions}
\emph{Transactional Synchronization Extensions} (TSX) é uma implementação de
transações de memória, que agrupa instruções em transacões executadas de modo
atômico, executando todas as instruções da transação especulativamente e
consolidando os resultados apenas após a execução de sua última instrução. Se
alguma instrução causa uma \emph{memory fault}, a transação inteira é
descartada.~\cite{schaik2020cacheout}

\subsubsection{TSX Asynchronous Abort}
\emph{TSX Asynchronous Abort} é um tipo de ataque que zera linhas de cache
antes de uma transação que carrega dados dessas linhas, causando uma falha
na transação. Alocando espaço no LFB antes da transação, os dados do LFB são
encaminhados para a instrução que causa a \emph{fault}. Como a transação não
é completada, a instrução é executada com dados de uma transação anterior,
permitindo a amostragem desses dados.~\cite{schaik2020cacheout}

\subsection{SGAxe} % lucas
\subsection{ZombieLoad} % julia
\subsection{Foreshadow} % julia

\newpage

%%%%%%%%%%%%%%%%%%%%%%%%%%%%%%%%%%%%%%%%%%%%%%%%%%%%%%%%%%%%%%%%%%%%%%%%%%%%%

\section{Há jeito de se previmir?}

\newpage

%%%%%%%%%%%%%%%%%%%%%%%%%%%%%%%%%%%%%%%%%%%%%%%%%%%%%%%%%%%%%%%%%%%%%%%%%%%%%

\section{Há como saber se estou sofrendo um SCA?}

No. =(=

\newpage

%%%%%%%%%%%%%%%%%%%%%%%%%%%%%%%%%%%%%%%%%%%%%%%%%%%%%%%%%%%%%%%%%%%%%%%%%%%%%

\section{Conclusão}

\newpage

\printbibliography

%%%%%%%%%%%%%%%%%%%%%%%%%%%%%%%%%%%%%%%%%%%%%%%%%%%%%%%%%%%%%%%%%%%%%%%%%%%%%

\end{document}
