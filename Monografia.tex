%%%%%%%%%%%%%%%%%%%%%%%%%%%%%%%%%%%%%%%%%%%%%%%%%%%%%%%%%%%%%%%%%%%%%%%%%%%%%

\documentclass[12pt, a4paper, oneside]{article}

\input{packages.tex}
\usepackage{csquotes}
\input{macros.tex}

\title{Side-Channel Attack}
\author{
    Leonardo Costa Santos - 10783142\\
    Lucas Paiolla Forastiere - 11221911\\
    Julia Leite - 11221797
}
\date{\today}

%%%%%%%%%%%%%%%%%%%%%%%%%%%%%%%%%%%%%%%%%%%%%%%%%%%%%%%%%%%%%%%%%%%%%%%%%%%%%
%%%%%%%%%%%%%%%%%%%%%%%%%%%%%%%%%%%%%%%%%%%%%%%%%%%%%%%%%%%%%%%%%%%%%%%%%%%%%

\begin{document}

\maketitle
\newpage 

\section{Introdução}

\par Side-Channel Attacks, que do inglês significa, Ataque por Canal Lateral são
jeitos explorar vunerabilidades físicas de componentes eletrônicos, como a CPU
de um computador.

O nome vem do fato deles não atacarem "pela porta da frente", mas sim algum
"rastro" físico que um componente deixa ao fazer determinadas ações.

Um SCA não necessariamente tem a ver com componentes eletrônicos, pois podemos,
por exemplo, descifrar uma senha de alguém captando os sons do teclado. Em
geral, o SCA ataca um ponto fraco de um componente que não tem nada a ver com o
seu funcionamento em si (como no exemplo do teclado, o teclado teoricamente não
tem a responsabilidade de deixar os barulhos de cada tecla iguaizinhos). Daí
então o nome \textit{canal lateral}.
% TODO: Explicar melhor o que é side channel <18-11-20, Lucas> %

Os dois principais SCA, que tornaram o ``ramo'' famoso foram o \textit{Meltdown}
e o \textit{Spectre}, descobertos por independentemente por uma série de
pesquisadores, mas destacando-se o grupo Project Zero da
Google~\cite{Lipp2018meltdown}.

Entretanto, existem muitas classes de SCA, como ataques ao cache (que é o caso
dos dois exemplos citados), ataques que monitoram a energia consumida pelo
computador, ataques que monitoram o eletromagnetismo eminido, ataques que
monitoram o som emitido (como o exemplo do teclado), ataques que recuperam dados
excluidos do disco entre muitos outros.

\newpage

%%%%%%%%%%%%%%%%%%%%%%%%%%%%%%%%%%%%%%%%%%%%%%%%%%%%%%%%%%%%%%%%%%%%%%%%%%%%%

\section{História} % julia

\par Um dos mais notórios ataques análogo ao que hoje conhecemos como ataque por canais laterais, foi chamado, 
pela \textit{National Security Agency} (NSA) de Tempest\cite{tempestNSA} \cite{historyWired},
e ocorreu em 1943, quando pesquisadores do \textit{Bell Lab} 
descobriram que, utilizando um osciloscópio, era possível recuperar $75\%$ do texto da mensagem 
emitita por um \textit{teletype machine} há 80 pés (aproximadamente 24.3m) de distância.

\par Entretanto, o termo \textit{Side Channel Cryptanalysis} sugiu 
apenas em 1998, em um artigo \cite{historyPaper} onde uma equipe formada por criptógrafos
da Counterpane Systems e pesquisadores da \textit{Universy of California
at Berkley} descreveu como ataques por canais laterais podem ser usados para 
quebrar sistemas de criptografia.

\par Em 2014, pesquisadores da \textit{Tel Aviv University} \cite{RatioTelAviv}
desenvolveram um dispositivo capaz de descobrir senhas criptografadas
de laptop's próximos motitorando sua emissão elétrica. Além disso, em agosto do mesmo
ano, esse grupo de pesquisadores publicou um artigo mostrando que o padrão 
sons emitidos por um computador durante o processo de descriptografia, detectável
por um microfone, varia de acordo com a chave RSA utilizada, apesar disso,
não ficou claro como extrair os bits da chave RSA.\cite{historyMic}

\par Desde 2018, o termo \textit{side channel attack} se popularizou
graças à descoberta de vunerabilidades presentes na maioria dos 
processadores Intel fabricados nas últimas duas décadas, exploradas
por SCA como Meltdown, Spectre, entre outros.

\newpage

%%%%%%%%%%%%%%%%%%%%%%%%%%%%%%%%%%%%%%%%%%%%%%%%%%%%%%%%%%%%%%%%%%%%%%%%%%%%%

\section{Classificações de Side-Channel Attacks} % lucas


\subsection{Classificação por (?)} % se é passivo ou ativo

\subsection{Classificação por grau da invasão} % residual integrity (melhorar dps)

%%%%%%%%%%%%%%%%%%%%%%%%%%%%%%%%%%%%%%%%%%%%%%%%%%%%%%%%%%%%%%%%%%%%%%%%%%%%%

\section{Exemplos de Side-Channel Attacks}

\subsection{Meltdown} % lucas

Teste citando \cite{Lipp2018meltdown}
\subsection{Spectre} % leo
\subsection{CacheOut} % leo
\subsection{SGAxe} % lucas


\subsection{ZombieLoad} % julia

\par Em 14 de maio de 2019 um grupo formado por pesquisadores de 
diversas universidades, dentre elas a austríaca  Graz University of 
Technology e a belga Catholic University of Leuven, e por equipes das 
empresas de segurança Oracle, Cyberus, entre outras, juntamente com a 
Intel, reportou uma novo tipo de SCA, nomeado ZombieLoad \cite{Schwarz2019ZombieLoad}, que explora 
vulnerabilidades presentes na maioria dos processadores Intel 
fabricados após 2011.\cite{zombieWired}
 
\par O ZombieLoad, assim com o Spectre, Meltdown, Foreshadow, entre 
outros, pertence à classe dos \textit{transient-execution attacks} \cite{canella2019systematic}, 
ataques que exploram vulnerabilidades resultantes de técnicas 
utilizadas para melhorar a performance do computador, como a execução 
fora de ordem e a execução especulativa. A primeira é um  paradigma 
que permite que a CPU, ao invés de executar as microoperações de um conjunto de 
instruções sequencialmente, execute-as em paralelo, mesmo que a 
microoperação anterior não tenha sido finalizada, e as reorganize 
depois, decidindo aproveitar ou descartar os resultados. 
Já a segunda  consiste na CPU executar instruções especulativamente, 
ou seja, antes delas aparecerem na lista de instruções, utilizando 
análise do fluxo de dados e predicção de desvio, assim como na técnica 
anterior, os resultados das instruções executadas especulativamente 
podem ser aproveitados ou descartados.

\par Nesse caso, chamamos essa instrução executada fora de ordem ou 
especulativamente, cujo resultado foi descartado de \textit{transient 
instruction}. Os efeitos da \textit{transient execution} são descartados, 
contudo, utilizando canais laterais como o \textit{CPU cache subsystem},
é possível extrair dados de outros processos carregados no 
mesmo core da CPU, como senhas, tokens, histórico de navegação do 
browser, entre outros.

\par A Intel liberou correções em microcódigo para os processadores 
vulneráveis e a 8ª e 9ª geração de processadores possuem correção em 
hardware. Contudo, essas mitigações reduzem a velocidade do computador 
em $3\%$ e a potência em $9\%$. Os pesquisadores, no entanto, afirmam 
que essas medidas são insuficientes para evitar que um computador 
esteja vulnerável a esse ataque e que a solução mais segura seria 
desabilitar o \textit{hyperthreading}. \cite{cyberus}


\subsection{Foreshadow} % julia

\newpage

%%%%%%%%%%%%%%%%%%%%%%%%%%%%%%%%%%%%%%%%%%%%%%%%%%%%%%%%%%%%%%%%%%%%%%%%%%%%%

\section{Há jeito de se previmir?}

\newpage

%%%%%%%%%%%%%%%%%%%%%%%%%%%%%%%%%%%%%%%%%%%%%%%%%%%%%%%%%%%%%%%%%%%%%%%%%%%%%

\section{Há como saber se estou sofrendo um SCA?}

No. =(=

\newpage

%%%%%%%%%%%%%%%%%%%%%%%%%%%%%%%%%%%%%%%%%%%%%%%%%%%%%%%%%%%%%%%%%%%%%%%%%%%%%

\section{Conclusão}

\newpage

\printbibliography

%%%%%%%%%%%%%%%%%%%%%%%%%%%%%%%%%%%%%%%%%%%%%%%%%%%%%%%%%%%%%%%%%%%%%%%%%%%%%

\end{document}
