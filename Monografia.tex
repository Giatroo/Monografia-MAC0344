%%%%%%%%%%%%%%%%%%%%%%%%%%%%%%%%%%%%%%%%%%%%%%%%%%%%%%%%%%%%%%%%%%%%%%%%%%%%%

\documentclass[12pt, a4paper, oneside]{article}

\input{packages.tex}
\usepackage{csquotes}
\input{macros.tex}

\title{Side-Channel Attack}
\author{
    Leonardo Costa Santos - 10783142 &&
    Lucas Paiolla Forastiere - 11221911 &&
    Julia Leite - 11221797
}
\date{\today}

%%%%%%%%%%%%%%%%%%%%%%%%%%%%%%%%%%%%%%%%%%%%%%%%%%%%%%%%%%%%%%%%%%%%%%%%%%%%%
%%%%%%%%%%%%%%%%%%%%%%%%%%%%%%%%%%%%%%%%%%%%%%%%%%%%%%%%%%%%%%%%%%%%%%%%%%%%%

\begin{document}

\maketitle
\newpage

\section{Introdução}

Side-Channel Attacks, que do inglês significa, Ataque por Canal Lateral são
jeitos explorar vunerabilidades físicas de componentes eletrônicos, como a CPU
de um computador.

O nome vem do fato deles não atacarem "pela porta da frente", mas sim algum
"rastro" físico que um componente deixa ao fazer determinadas ações.

Um SCA não necessariamente tem a ver com componentes eletrônicos, pois podemos,
por exemplo, descifrar uma senha de alguém captando os sons do teclado. Em
geral, o SCA ataca um ponto fraco de um componente que não tem nada a ver com o
seu funcionamento em si (como no exemplo do teclado, o teclado teoricamente não
tem a responsabilidade de deixar os barulhos de cada tecla iguaizinhos). Daí
então o nome \textit{canal lateral}.
% TODO: Explicar melhor o que é side channel <18-11-20, Lucas> %

Os dois principais SCA, que tornaram o ``ramo'' famoso foram o \textit{Meltdown}
e o \textit{Spectre}, descobertos por independentemente por uma série de
pesquisadores, mas destacando-se o grupo Project Zero da
Google~\cite{Lipp2018meltdown}.

Entretanto, existem muitas classes de SCA, como ataques ao cache (que é o caso
dos dois exemplos citados), ataques que monitoram a energia consumida pelo
computador, ataques que monitoram o eletromagnetismo eminido, ataques que
monitoram o som emitido (como o exemplo do teclado), ataques que recuperam dados
excluidos do disco entre muitos outros.

\newpage

%%%%%%%%%%%%%%%%%%%%%%%%%%%%%%%%%%%%%%%%%%%%%%%%%%%%%%%%%%%%%%%%%%%%%%%%%%%%%

\section{História} % julia

Primeira pessoa a usar o termo (Side Channel Cryptanalysis of Product Ciphers)
Primeiro ataque descoberto (TEMPEST 1942)
Popularização do termo
Mais descobertas de ataques

\newpage

%%%%%%%%%%%%%%%%%%%%%%%%%%%%%%%%%%%%%%%%%%%%%%%%%%%%%%%%%%%%%%%%%%%%%%%%%%%%%

\section{Classificações de Side-Channel Attacks} % lucas


\subsection{Classificação por (?)} % se é passivo ou ativo

\subsection{Classificação por grau da invasão} % residual integrity (melhorar dps)

%%%%%%%%%%%%%%%%%%%%%%%%%%%%%%%%%%%%%%%%%%%%%%%%%%%%%%%%%%%%%%%%%%%%%%%%%%%%%

\section{Exemplos de Side-Channel Attacks}

\subsection{Meltdown} % lucas

Meltdown é um SCA relacionado ao processador e a um efeito colateral da execução
fora-de-ordem feita por ele. Graças a ela, o Meltdown consegue quebrar a
hierarquia entre o \textit{espaço do usuário} e o \textit{espaço do núcleo},
podendo ler informações que não deveriam ser acessíveis por qualquer usuário,
como senhas e dados pessoais.

Atualmente, o principal mecanismo de defesa de qualquer sistema operacional é a
\textit{isolação da memória}, dividindo a memória principal entre os diversos
processos em andamento de forma que as regiões de memória em uso por um não
possam ser acessadas por outros.

Para conseguir isso e outras propriedades importantes do Sistema Operacional,
ele se utiliza da chamada \textit{memória virtual}, que é uma abstração para a
memória física. Essa memória virtual é divide em páginas de memória que podem
ser individualmente mapeadas em regiões da memória física através de uma
\textit{tabela de tradução de páginas}.

Essa tabela não só tem como função mapear as páginas de memória, mas também
dividi-las entre as páginas que pertecem ao usuário e às que pertencem ao núcleo
(e daí surgem os termos \textit{espaço do usuário} e \textit{espaço do núcleo}).

Através da tabela, o SO garante que o usuário não conseguirá acessar espaços de
acesso restrito ao núcleo. Entretanto, o núcleo pode e deve ter acesso a toda a
memória física em si (inclusive a parte em que se mapeam as páginas de usuário).
Isso significa na prática que dentro da memória virtual do núcleo, existe uma
região que mapea toda a memória física, permitindo que o núcleo altere posições
de memória do usuário quando ele faz chamadas ao sistema.

\begin{figure}[h]
\centering
\includegraphics[scale=0.9]{Res/virtual_memory_mapping.png}
\caption{A memória física é completamente mapeada pelo núcleo em um certo ponto
da sua memória virtual. Os endreços físicos (em azul) são mapeados pelo usuário,
mas também pelo núcleo, sendo acessível pelos dois com eficiência.}
\label{virtual_memory_mapping}
\end{figure}

Além disso, outro mecanismo crucial de defesa é a separação entre o
\textit{espaço do usuário} e o \textit{espaço do núcleo}, ou seja, uma divisão
entre quais processos (e quais partes da memória) pertencem a processos do
usuário e quais pertencem ao sistema operacional em si.

Essa divisão é tipicamente realizada por um bit supervisor do processador que
define se uma página de memória do núcleo pode ou não ser acessada. E a ideia
por trás desse bit é que ele será mudado para $1$ quando um processo precisa
fazer chamadas ao sistemas (\textit{syscalls}), portanto parando de executar
código do usuário e passando a executar código do núcleo, e mudado novamente
para $0$ quando saímos do modo núcleo.

Esse bit é utilizado com uma ideia de eficiência, pois permite que o SO mapeie o
núcleo no espaço de endereço do processo que fez a \textit{syscall}.
Consequentemente, não há nenhuma mudança no mapeamento da memória quando mudamos
do modo usuário para o modo núcleo (o que é crucial para garantir mais
velocidade).

O Meltdown explora uma vunerabilidade causa pela \textit{execução fora-de-ordem}
para ler dados mapeados no espaço de endereço do núcleo, o que inclui a memória
física inteira em sistemas Linux, Android e OS X e uma grande parte da memória
física em ambientes Windows.

As CPUs modernas possuem essa técnica de otimização chamada de \textit{execução
fora-de-ordem} que permite que os núcleos passem mais tempo trabalhando, mesmo
quando uma determinada operação precisa esperar algum recurso (por exemplo,
trazer um valor da memória). Basicamente, o que acontece é que ao invés de a CPU
executar as instruções sequenciamente, ela vai as executando assim que todos os
recursos necessários para uma determinada instrução estiverem disponíveis.

Na prática, isso significa que a CPU \textit{especula} que uma determina
instrução será executada no futuro e, então, faz a sua execução antes mesmo de
ter certeza disso. O desenvolvedor do algoritmo que possibilitou a
\textit{execução fora-de-ordem} foi Tomasulo em
1967~\cite{Tomasulo1967outoforder}.

% TODO: Pensar se explica como é implementada a execução fora de ordem em um
% procesador Intel. Daí colocar a foto igual no artigo <24-11-20, Lucas> %

A vunerabilidade encontrada pelo Meltdown se deve ao fato de que ao tentar
executar uma instrução de acessar uma região de memória que não pertence ao
programa, a \textit{execução fora-de-ordem} acabará fazendo o acesso e
armazenando o valor no \textit{cache}. Apenas depois que esse dado é armazenado
no \textit{cache}, a CPU percebe que a instrução não deveria ser executada e
não de fato entrega esse valor ao programa que solicitou. Contudo, como o dado
está em cache, o programa pode fazer um \textit{ataque ao cache} para recuperar
essa informação, acessando, portanto, uma região da memória que não pertence ao
programa atacante.

\subsection{Spectre} % leo
\subsection{CacheOut} % leo
\subsection{SGAxe} % lucas
\subsection{ZombieLoad} % julia
\subsection{Foreshadow} % julia

\newpage

%%%%%%%%%%%%%%%%%%%%%%%%%%%%%%%%%%%%%%%%%%%%%%%%%%%%%%%%%%%%%%%%%%%%%%%%%%%%%

\section{Há jeito de se previmir?}

\newpage

%%%%%%%%%%%%%%%%%%%%%%%%%%%%%%%%%%%%%%%%%%%%%%%%%%%%%%%%%%%%%%%%%%%%%%%%%%%%%

\section{Há como saber se estou sofrendo um SCA?}

No. =(=

\newpage

%%%%%%%%%%%%%%%%%%%%%%%%%%%%%%%%%%%%%%%%%%%%%%%%%%%%%%%%%%%%%%%%%%%%%%%%%%%%%

\section{Conclusão}

\newpage

\printbibliography

%%%%%%%%%%%%%%%%%%%%%%%%%%%%%%%%%%%%%%%%%%%%%%%%%%%%%%%%%%%%%%%%%%%%%%%%%%%%%

\end{document}
