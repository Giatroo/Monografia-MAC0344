%%%%%%%%%%%%%%%%%%%%%%%%%%%%%%%%%%%%%%%%%%%%%%%%%%%%%%%%%%%%%%%%%%%%%%%%%%%%%

\documentclass[12pt, a4paper, oneside]{article}

\input{packages.tex}
\usepackage{csquotes}
\input{macros.tex}

\title{Side-Channel Attack}
\author{
    Leonardo Costa Santos - 10783142 &&
    Lucas Paiolla Forastiere - 11221911 &&
    Julia Leite - 11221797
}
\date{\today}

%%%%%%%%%%%%%%%%%%%%%%%%%%%%%%%%%%%%%%%%%%%%%%%%%%%%%%%%%%%%%%%%%%%%%%%%%%%%%
%%%%%%%%%%%%%%%%%%%%%%%%%%%%%%%%%%%%%%%%%%%%%%%%%%%%%%%%%%%%%%%%%%%%%%%%%%%%%

\begin{document}

\maketitle
\newpage

\section{Introdução}

Side-Channel Attacks, que do inglês significa, Ataque por Canal Lateral são
jeitos explorar vunerabilidades físicas de componentes eletrônicos, como a CPU
de um computador.

O nome vem do fato deles não atacarem "pela porta da frente", mas sim algum
"rastro" físico que um componente deixa ao fazer determinadas ações.

Um SCA não necessariamente tem a ver com componentes eletrônicos, pois podemos,
por exemplo, descifrar uma senha de alguém captando os sons do teclado. Em
geral, o SCA ataca um ponto fraco de um componente que não tem nada a ver com o
seu funcionamento em si (como no exemplo do teclado, o teclado teoricamente não
tem a responsabilidade de deixar os barulhos de cada tecla iguaizinhos). Daí
então o nome \textit{canal lateral}.
% TODO: Explicar melhor o que é side channel <18-11-20, Lucas> %

Os dois principais SCA, que tornaram o ``ramo'' famoso foram o \textit{Meltdown}
e o \textit{Spectre}, descobertos por independentemente por uma série de
pesquisadores, mas destacando-se o grupo Project Zero da
Google~\cite{Lipp2018meltdown}.

Entretanto, existem muitas classes de SCA, como ataques ao cache (que é o caso
dos dois exemplos citados), ataques que monitoram a energia consumida pelo
computador, ataques que monitoram o eletromagnetismo eminido, ataques que
monitoram o som emitido (como o exemplo do teclado), ataques que recuperam dados
excluidos do disco entre muitos outros.

\newpage

%%%%%%%%%%%%%%%%%%%%%%%%%%%%%%%%%%%%%%%%%%%%%%%%%%%%%%%%%%%%%%%%%%%%%%%%%%%%%

\section{História} % julia

Primeira pessoa a usar o termo (Side Channel Cryptanalysis of Product Ciphers)
Primeiro ataque descoberto (TEMPEST 1942)
Popularização do termo
Mais descobertas de ataques

\newpage

%%%%%%%%%%%%%%%%%%%%%%%%%%%%%%%%%%%%%%%%%%%%%%%%%%%%%%%%%%%%%%%%%%%%%%%%%%%%%

\section{Classificações de Side-Channel Attacks} % lucas


\subsection{Classificação por (?)} % se é passivo ou ativo

\subsection{Classificação por grau da invasão} % residual integrity (melhorar dps)

%%%%%%%%%%%%%%%%%%%%%%%%%%%%%%%%%%%%%%%%%%%%%%%%%%%%%%%%%%%%%%%%%%%%%%%%%%%%%

\section{Exemplos de Side-Channel Attacks}

\subsection{Meltdown} % lucas

Teste citando \cite{Lipp2018meltdown}
\subsection{Spectre} % leo
\subsection{CacheOut} % leo
\subsection{SGAxe} % lucas
\subsection{ZombieLoad} % julia
\subsection{Foreshadow} % julia

\newpage

%%%%%%%%%%%%%%%%%%%%%%%%%%%%%%%%%%%%%%%%%%%%%%%%%%%%%%%%%%%%%%%%%%%%%%%%%%%%%

\section{Há jeito de se previmir?}

\newpage

%%%%%%%%%%%%%%%%%%%%%%%%%%%%%%%%%%%%%%%%%%%%%%%%%%%%%%%%%%%%%%%%%%%%%%%%%%%%%

\section{Há como saber se estou sofrendo um SCA?}

No. =(=

\newpage

%%%%%%%%%%%%%%%%%%%%%%%%%%%%%%%%%%%%%%%%%%%%%%%%%%%%%%%%%%%%%%%%%%%%%%%%%%%%%

\section{Conclusão}

\newpage

\printbibliography

%%%%%%%%%%%%%%%%%%%%%%%%%%%%%%%%%%%%%%%%%%%%%%%%%%%%%%%%%%%%%%%%%%%%%%%%%%%%%

\end{document}
